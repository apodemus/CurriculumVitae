% LaTeX resume using res.cls
\listfiles
\documentclass{article}

\usepackage[T1]{fontenc}
\usepackage[utf8]{inputenc}
% \usepackage{kpfonts} % lmodern
\usepackage{bold-extra}  % It works by building a bold Computer Modern small-caps font on the fly and including it for you
% \normalfont %to load T1lmr.fd 
% \DeclareFontShape{T1}{lmr}{bx}{sc} { <-> ssub * cmr/bx/sc }{}

\usepackage{hyperref}
\hypersetup{
     colorlinks=true,
%     linkcolor=blue,
%     filecolor=magenta,      
    urlcolor=blue,
}
%\usepackage{amssymb}
% \usepackage{helvetica} % uses helvetica postscript font (download helvetica.sty)
%\usepackage{newcent}   % uses new century schoolbook postscript font
\usepackage{tabularx}
\usepackage{fancyhdr}  % use fancyhdr package to get 2-line header

\usepackage{titlesec}
% from the titlesec package
%\titleformat{ command }
%             [ shape ]
%             { format }{ label }{ sep }{ before-code }[ after-code ]
% \section*

\titleformat{\section}[block]
  {\filcenter\large\bfseries\scshape}%% \normalfont since CMR doesn't have bold small caps. If you use some other font like libertine then you can have bold small caps
  {}
  {0pt}
  {\titlerule\\}[\vspace{2pt}\titlerule]

% \newcommand{\sectionfont}{\Large\bfseries\scshape}
% 
% \titleformat{\section}[block]
%     {\titlerule
%      \vspace{0.5ex}%
%      \sectionfont}
%     {\thesection}{1em}
%     {\sectionfont}[\titlerule]
    
% \titleformat{name=\section,numberless}
% {\filcenter\Large\bfseries}
% {}
% {.5em}
% {}

% Custom section headers
% TODO: the section headers are now not registering in the pdf, so can't jump around with links.  FIXME
% \renewcommand{\section}[1]{%
%   \vspace{0.4cm}%
%   \begin{table}[!htp]%					% not h on its own
%     \newcolumntype{C}{>{\centering\arraybackslash}X}%	% centering
%     \setlength\extrarowheight{3pt}% 			% extra padding
%     \noindent% 						% otherwise the line will be too wide by \parindent
%     \begin{tabularx}{\textwidth}{C}%
%       \hline \hline%
% 	\large \textbf{\textsc{#1}} \\%
%       \hline \hline%
%     \end{tabularx}
%   \end{table}
% }


\usepackage[backend=biber,
	    style=trad-abbrv,%numeric
	    sorting=ydnt, % sort by date
	    doi=false,
	    eprint=false]{biblatex}
\addbibresource{publications/articles.bib}
\addbibresource{publications/proceedings.bib}
\addbibresource{publications/shortpub.bib}


\usepackage[margin=0.75in]{geometry}
\setlength{\topmargin}{-0.6in}  % Start text higher on the page 
\setlength{\textheight}{9.5in}  % increase textheight to fit more on a page
\setlength{\headsep}{0.2in}     % space between header and text
\setlength{\headheight}{12pt}   % make room for header
\setlength{\parindent}{0cm}


%\renewcommand{\headrulewidth}{0pt} % suppress line drawn by default by fancyhdr
\lhead{Hannes Breytenbach} % \hspace*{-\sectionwidth} % force lhead all the way left
\rhead{Page \thepage}  % put page number at right
\cfoot{}  % the footer is empty
\pagestyle{fancy} % set pagestyle for the document

% \usepackage{titlesec}
% \titleformat{\section}
%   {\leaders\vrule width \textwidth\vskip 0.4pt\vspace{.8ex}%
%    \normalfont\Large\bfseries}
%   {}
%   {0em}
%   {}
  
% \titleformat{\section}[display]
% {\normalfont\large\bfseries}{\chaptertitlename\ \thechapter}{20pt}{\Huge}[\vspace{2ex}\titlerule]

\renewcommand{\arraystretch}{1.2}


\begin{document} 
\thispagestyle{empty} % this page does not have a header

%%%%%%%%%%%%%%%%%%%%%%%%%%%%%%%%%%%%%%%%%%%%%%%%%%%%%%%%%%%%%%%%%%%%%%%%%%%%%%%%%%%%%%% HEADING %%%%%%%%%%%%%%%%%%%%%%%%%%%%%%%%%%%%%%%%%%%%%%%%%%%%%%%%%%%%%%%%%%%%%%%%%%%%%%%%%%%%%%%
\begin{center}
  \large\emph{Curriculum Vitae}\\
  \vspace{0.4cm}
  \huge \textbf{\textsc{Hannes Breytenbach}}\\
  \vspace{0.2cm}
  \normalsize
  % \emph{16 Hyde Park Apartments \textbullet\ 14 Jetty Street \textbullet\ Foreshore \textbullet\ Cape Town \textbullet\ 8001}\\
  \emph{email: hannes@ saao.ac.za}\\
  \emph{Cell: +27 82 726 9311}\\

\end{center}

%%%%%%%%%%%%%%%%%%%%%%%%%%%%%%%%%%%%%%%%%%%%%%%%%%%%%%%%%%%%%%%%%%%%%%%%%%%%%%%%%%%%%%% INFO %%%%%%%%%%%%%%%%%%%%%%%%%%%%%%%%%%%%%%%%%%%%%%%%%%%%%%%%%%%%%%%%%%%%%%%%%%%%%%%%%%%%%%%
\section*{Personal Information}
\begin{tabular}[h!]{l l} 
  \textbf{Names:}		& Johannes Benjamin Breytenbach\\
  \textbf{Date of Birth:}	& 26 June 1987	\\
  \textbf{Place of Birth:}	& Santiago, Chile\\
  \textbf{Citizenship:}		& South African\\
  \textbf{Languages:}		& English, Afrikaans (Highly Proficient); \\
				& German (Intermediate) \\
  \textbf{Current Occupation:}	& PhD. Student at\\&\hspace{0.5cm} South African Astronomical Observatory (SAAO) \& \\&\hspace{0.5cm} University of Cape Town, (UCT)\\
  \textbf{Work Address:}	& Room 4-31, RW James Building \\&9 University Avenue \\&Upper Campus, UCT %\\&Woolsack Drive 
				  \\&Rondebosch, 7001 \\&Cape Town \\&South Africa
\end{tabular}

% % 
\section*{Education}
\begin{tabular}{lll}
  \textbf{2011 - present:} 	& UCT - PhD (Astrophysics)\\
    \multicolumn{3}{c}{\parbox{0.95\textwidth}{
	\begin{tabular}{p{0.175\linewidth}p{0.8\linewidth}}
	  \textbullet\ Thesis title: 	& \emph{``Rapid Variability in magnetic Cataclysmic Variable Stars''} \\
	  \textbullet\ Supervisors:	& Dr. David Buckley, A. Prof Patrick Woudt \\
	  \textbullet\ Modules: 	& Cataclysmic Variable Stars, Stellar Structures, Advanced General Relativity, Hot Topics in Cosmology, High Energy Astrophysics \\ 
	\end{tabular} }}
      \\ \\
%   
  \textbf{2010:} 		& UCT - BSc Honours (Astrophysics and Space Science) \\
    \multicolumn{3}{c}{\parbox{0.95\textwidth}{
	\begin{tabular}{p{0.175\linewidth}p{0.8\linewidth}}
	  \textbullet\ Project title:	& \emph{``The Sferic Count Rate from SANAE-IV, Antarctica''} \\
	  \textbullet\ Supervisors:	& Dr. Andrew Collier	\\
	  \textbullet\ Modules: 	& General Astrophysics, Electrodynamics, General Relativity, Computational Astrophysics, Galaxies and Large Scale Structure, Observational Techniques, Radio Astronomy\\ 
	\end{tabular} }}
      \\ \\
%   
  \textbf{2006 - 2009:} 		& University of Pretoria (UP)-  BSc (Physics and Astronomy) \\ %\ae\
    \multicolumn{3}{c}{\parbox{0.95\textwidth}{
	\begin{tabular}{p{0.175\linewidth}p{0.8\linewidth}}
	  \textbullet\ Project title:	& \emph{``Rutherford Backscattering Spectroscopy and X-ray Diffraction Spectroscopy of Aluminium 100''} \\
	  \textbullet\ Supervisors:	& Prof. Chris Theron\\
	  \textbullet\ Modules: 	& Quantum Mechanics, Solid State Physics, Statistical Mechanics, Differential Calculus, Vector Calculus, Partial Differential Equations, Abstract Algebra, Mathematical Modelling\\
	\end{tabular} }}
      \\
%       
 \end{tabular}
% 
% 
\section*{Awards and Scholarships}
\begin{tabular}{p{0.175\linewidth} p{0.8\linewidth}}
  \textbullet\ 2013 - 2016: 		& NRF, Postgraduate Development Programme (PDP) Doctoral Scholarship \\
  \textbullet\ 2010 - 2012: 		& South African Square Kilometre Array (SKA) Postgraduate Scholarship \\
  \textbullet\ 2008 - 2009: 		& SKA Undergraduate Bursary Award\\
\end{tabular}
% 
\section*{Special Skills}

\large\textbf{Data \& Computational skills}
\normalsize
\vspace{0.2cm}


\begin{itemize}
  \item Public code repository: \hspace{1.7cm} \url{https://github.com/apodemus}
  
  \item Languages:\\
  \begin{tabular}{p{0.5\linewidth}p{0.8\linewidth}}
    \textbullet\ Python					& (Expert) \\
    \textbullet\ IRAF, MATLAB (Octave)			& (Highly proficient) \\
    \textbullet\ IDL, R, C, Mathematica, Maple, LabView & (Intermediate) \\
    \textbullet\ SQL 					& (Novice)\\
  \end{tabular}

  \item Algorithm Development:
  \begin{itemize}
    \item Machine learning algorithms for classification of EEG data (epileptic seizure prediction)
    \item Time series analysis \& Spectral estimation techniques (Ph.D. Thesis)
    \item Data reduction pipeline for Sutherland High Speed Optical Camera (SHOC)
    \item Computational modelling of Dwarf Novae in Outburst (M.Sc. Thesis)
  \end{itemize}
\end{itemize}

\vspace{0.2cm}
\large\textbf{Observing}
\normalsize
\vspace{0.2cm}

\begin{itemize}
 \item 98 nights on 1.9m telescope at SAAO, Sutherland\\
  \hspace*{0.4cm} - Rapid photometry of CVs (SHOC CCD)\\ %(84 nights) %7
  \hspace*{0.4cm} - Polarimetry of magnetic CVs (HIPPO)\\% (Assisting) %7
  \hspace*{0.4cm} - Spectroscopy of Dwarf Stars %((SkyMapper follow-up)\\
 
 \item 21 nights on 1.0m telescope at SAAO, Sutherland \\
  \hspace*{0.4cm} -  Multi-colour photometry of interacting galaxies \\
  \hspace*{0.4cm} -  High speed photometry of CVs
 
 \item 7 nights on 1.4m Infra-red Survey Facility telescope (IRSF) at SAAO, Sutherland \\
 \hspace*{0.4cm} - Photometry of interacting galaxies (during ISYA)
\end{itemize}  
  
  


\section*{Professional Development}

%TODO: links to online presentations where available

\large\textbf{Conference Attendance}
\vspace{0.2cm}
\normalsize
\begin{itemize}
 \item 2016 July: South African Institute of Physics (SAIP2016), UCT\\
 \hspace*{0.4cm} - Presentation: ``Quasi-Periodic Oscillations in magnetic CVs''
 \item 2015 September: The Golden Age of Cataclysmic Variables and Related Objects - III\\
 \hspace*{0.4cm} - Presentation: ``Quasi-Periodic Oscillations in magnetic CVs''
 \item 2015 June: SALT Science Conference 2015, STIAS, Stellenbosch \\
 \hspace*{0.4cm} - Poster: ``Probing accretion in magnetic CVs through rapid photometry with SALTICAM''
 \item 2014 July: South African Institute of Physics Conference (SAIP2014)\\
 \hspace*{0.4cm} - Presentation: ``Rapid Variability of magnetic Cataclysmic Variable Stars''
 \item 2013 September: The Golden Age of Cataclysmic Variables and Related Objects - II\\
 \hspace*{0.4cm} - Presentation: ``Modelling Quasi-Periodic Variability in Cataclysmic Variable Stars''
 \item 2013 July: SKA Joint Radio Transients Conference\\
 \hspace*{0.4cm}
 \item 2012 December: SKA Postgraduate Bursary Conference\\
 \hspace*{0.4cm} - Presentation: ``Modelling Quasi-Periodic Variability in Cataclysmic Variable Stars''
 \item 2012 August: IAU XXVIII General Assembly\\
 \hspace*{0.4cm} - Poster: ``Modelling Quasi-Periodic Variability in Cataclysmic Variable Stars''
 \item 2011 December: SKA Postgraduate Bursary Conference\\
 \hspace*{0.4cm} - Poster: ``A study of DNOs and QPOs in Cataclysmic Variable Stars''
 \item 2011 March: Middle-East and African Regional IAU Meeting II (MEARIM-II)
 \item 2010 - 2008 November: SKA Postgraduate Bursary Conference
\end{itemize}
\vspace{0.4cm}
% 
\large\textbf{Workshop Attendance}
\vspace{0.2cm}
\normalsize
\\
\begin{tabular}{l p{0.8\linewidth}}
 \textbullet\ 2017 April: 	& Workshop on Magnetic Accretion, SAAO\\
 \textbullet\ 2017 April: 	& SKA Big Data Africa Summer School, Cape Town\\
 \textbullet\ 2016 November: 	& Workshop on Bayesian Analysis in Physics and Astronomy, Stellenbosch\\
 \textbullet\ 2016 May: 	& CDS Tools Workshop, SAAO\\
 \textbullet\ 2015 November: 	&  Workshop on using ALMA for science, UCT\\
 \textbullet\ 2015 April: 	& GPGPU programming workshop, UCT\\
 \textbullet\ 2014 November: 	& \textsc{GADGET} Simulations workshop, UCT\\
 \textbullet\ 2014 October: 	& $2^{nd}$ Machine learning JEDI Workshop, Cape Town\\
 \textbullet\ 2012 February: 	& IAU International School of Young Astronomers (ISYA), UCT \& SAAO, Cape Town\\
 \textbullet\ 2011 October: 	& Workshop on Space Science and Astrophysics, Centre for High Performance Computing (CHPC), CSIR, Pretoria \\
 \textbullet\ 2010 December: 	& Workshop on Convection in stars, University of the Witwatersrand (WITS), Johannesburg\\
 \textbullet\ 2010 January: 	& National Astrophysics and Space Science Program (NASSP) Summer School, UCT \& SAAO, Cape Town\\
\end{tabular}


% \begin{itemize}
%  \item 2017 April: Workshop on Magnetic Accretion, SAAO
%  \item 2017 April: SKA Big Data Africa Summer School, Cape Town
%  \item 2016 November: Workshop on Bayesian Analysis in Physics and Astronomy, Stellenbosch
%  \item 2016 May: CDS Tools Workshop, SAAO
%  \item 2015 November:  Workshop on using ALMA for science, UCT
%  \item 2015 April: GPGPU programming workshop, UCT
%  \item 2014 November: \textsc{GADGET} Simulations workshop, UCT
%  \item 2014 October: $2^{nd}$ Machine learning JEDI Workshop, Cape Town
%  \item 2012 February: IAU International School of Young Astronomers (ISYA), UCT \& SAAO, Cape Town
%  \item 2011 October: Workshop on Space Science and Astrophysics, Centre for High Performance Computing (CHPC), CSIR, Pretoria 
%  \item 2010 December: Workshop on Convection in stars, University of the Witwatersrand (WITS), Johannesburg
%  \item 2010 January: National Astrophysics and Space Science Program (NASSP) Summer School, UCT \& SAAO, Cape Town\\
% \end{itemize}


\vspace{1cm}
\section*{Publications}
% \vspace{0.2cm}
%TODO: bonus points if you can get the hyperlink to the url embedded in the printed text somehow

\begin{refsection}[publications/articles]
  \nocite{*}
  \printbibliography[title={Peer-reviewed},
		     heading=subbibliography]
\end{refsection}


\begin{refsection}[publications/proceedings]
  \nocite{*}
  \printbibliography[title={Conference Proceedings},
		     heading=subbibliography]
\end{refsection}


\begin{refsection}[publications/shortpub]
  \nocite{*}
  \printbibliography[title={Short publications},
		     heading=subbibliography]
\end{refsection}


\section*{Work and Teaching Experience}
% \vspace{0.2cm}
\begin{tabular}{l p{0.8\linewidth}}
  \textbullet\ 2013: &			Tutor: Stellar Structures (AST3002F), UCT \\
  \textbullet\ 2012: &			Tutor: Introductory Astronomy (AST2002S), UCT \\
  \textbullet\ 2012: &			Tutor: Introductory Astronomy (private) \\
  \textbullet\ 2011: &			Tutor: Electrodynamics Honours, NASSP, UCT \\
  \textbullet\ 2009: &			Tutor: Biological Physics (PHY103), $1^{st}$ year course, UP \\
  \textbullet\ 2007 Dec - Jan: &	Student Data Processor, Hartebeest Hoek Radio Astronomy Observatory (HartRAO)\\
\end{tabular}

% \begin{itemize}
%  \item 2013: Tutor: Stellar Structures (AST3002F), UCT
%  \item 2012: Tutor: Introductory Astronomy (AST2002S), UCT
%  \item 2012: Tutor: Introductory Astronomy (private)
%  \item 2011: Tutor: Electrodynamics Honours, NASSP, UCT
%  \item 2009: Tutor: Biological Physics (PHY103), $1^{st}$ year course, UP
%  \item 2007 Dec - Jan: Student Data Processor, Hartebeest Hoek Radio Astronomy Observatory (HartRAO)\\
% \end{itemize}



\section*{Research interests}
% \vspace{0.2cm}
\large\textbf{Academic}
\normalsize
  \begin{itemize}
    \item Astrophysical Accretion on all scales
    \item Compact binary stars
    \item Astrophysical transients
  \end{itemize}
\large\textbf{Techniques}
\normalsize
  \begin{itemize}
    \item Multi-wavelength astronomy
    \item Machine Learning
    \item Bayesian Modelling and Inference
%     \item Statistical Modelling
  \end{itemize}



\section*{Leadership \& Extracurricular Involvements}
\large\textbf{Academic}
\vspace{0.2cm}
\normalsize
\begin{itemize}
 \item 2017: Astronomy/Physics Dept. Mentorship programme volunteer
 \item 2014 - 2016: Postgraduate Student Representative, Astronomy Dept., UCT
 \item 2011: Siyavula Education Outreach Program
 \begin{itemize}
  \item[-] Tasks: Proof reading and Translation into Afrikaans of open source High School science textbooks
  \end{itemize}
\end{itemize} 
\vspace{0.2cm}

\large\textbf{Sport}
\vspace{0.2cm}
\normalsize
\begin{itemize}
  \item 2014: Chairperson: UCT Mountain and Ski Club
  \item 2011 - 2015: Committee member: UCT Mountain and Ski Club
  \item 2012: UCT Sports Merit Award
  \begin{itemize}
      \item[-] UIAA Youth expedition to summit the highest mountain in Europe, Mt Elbrus at 5 642m
  \end{itemize}
  \item 2011: UCT Sports Performance of the Year Award
  \begin{itemize}
      \item[-] UCT Mountain and Ski Club (MSC) mountaineering expedition to summit the 6 264m Himalayan peak, CB13-A	
  \end{itemize}
 
\end{itemize}


\section*{References}
\begin{itemize}
%  \item Emeritus Distinguished Professor Brian Warner\\Astronomy Department\\University of Cape Town\\Phone: +27 21 650 2391\\email: brian.warner@uct.ac.za
%  \item 

 \item Prof. Patrick Woudt\\Head of Department Astronomy\\ University of Cape Town\\Phone: +27 21 650 2392\\email: pwoudt@ast.uct.ac.za
 \item Dr. Nadeem Oozeer\\Commissioning scientist\\SKA SA\\Phone: +27 21 506 7325\\email: nadeem@ska.ac.za
 \item Dr. Bruce Bassett\\Senior Researcher\\African Institute for Mathematical Sciences\\email: bruce.a.bassett@gmail.com
 
 
\end{itemize}


\end{document}
