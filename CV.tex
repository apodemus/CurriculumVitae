% LaTeX resume using res.cls
\listfiles
\documentclass{article}

%\usepackage{helvetica} % uses helvetica postscript font (download helvetica.sty)
%\usepackage{newcent}   % uses new century schoolbook postscript font
\usepackage{tabularx}
\usepackage{fancyhdr}  % use fancyhdr package to get 2-line header
 \usepackage[T1]{fontenc}
%\usepackage{amssymb}
\usepackage[utf8]{inputenc}
\usepackage{hyperref}

\usepackage[backend=biber,
	    style=trad-abbrv]{biblatex}
\addbibresource{articles.bib}
\addbibresource{proceedings.bib}
\addbibresource{shortpub.bib}


\usepackage[margin=0.75in]{geometry}
\setlength{\topmargin}{-0.6in}  % Start text higher on the page 
\setlength{\textheight}{9.8in}  % increase textheight to fit more on a page
\setlength{\headsep}{0.2in}     % space between header and text
\setlength{\headheight}{12pt}   % make room for header
\setlength{\parindent}{0cm}


%\renewcommand{\headrulewidth}{0pt} % suppress line drawn by default by fancyhdr
\lhead{\hspace*{-\sectionwidth}Hannes Breytenbach} % force lhead all the way left
\rhead{Page \thepage}  % put page number at right
\cfoot{}  % the footer is empty
\pagestyle{fancy} % set pagestyle for the document



  
% Custom commands
\renewcommand{\section}[1]{
  \vspace{0.4cm}
  \begin{table}[!htp]					% not h on its own
    \newcolumntype{C}{>{\centering\arraybackslash}X}	% centering
    \setlength\extrarowheight{3pt} 			% extra padding

    \noindent 						% otherwise the line will be too wide by \parindent
    \begin{tabularx}{\textwidth}{C}
    \hline \hline 
      \large \textbf{ \textsc{#1} } \\ 
    \hline \hline
    \end{tabularx}
  \end{table}
}


%%%%%%%%%%%%%%%%%%%%%%%%%%%%%%%%%%%%%%%%%%%%%%%%%%%%%%%%%%%%%%%%%%%%%%%%%%%%%%%%%%%%%%% HEADING %%%%%%%%%%%%%%%%%%%%%%%%%%%%%%%%%%%%%%%%%%%%%%%%%%%%%%%%%%%%%%%%%%%%%%%%%%%%%%%%%%%%%%%
\begin{document} 
\thispagestyle{empty} % this page does not have a header

\begin{center}
  \large\emph{Curriculum Vitae}\\
  \vspace{0.4cm}
  \huge \textbf{\textsc{Hannes Breytenbach}}\\
  \vspace{0.2cm}
  \normalsize
  % \emph{16 Hyde Park Apartments \textbullet\ 14 Jetty Street \textbullet\ Foreshore \textbullet\ Cape Town \textbullet\ 8001}\\
  \emph{email: hannes@saao.ac.za}\\
  \emph{Cell: +27 82 726 9311}\\

\end{center}

%%%%%%%%%%%%%%%%%%%%%%%%%%%%%%%%%%%%%%%%%%%%%%%%%%%%%%%%%%%%%%%%%%%%%%%%%%%%%%%%%%%%%%% INFO %%%%%%%%%%%%%%%%%%%%%%%%%%%%%%%%%%%%%%%%%%%%%%%%%%%%%%%%%%%%%%%%%%%%%%%%%%%%%%%%%%%%%%%
% \begin{resume}

\section{Personal Information}
\begin{tabular}[h!]{l l} 
  \textbf{Names:}		& Johannes Benjamin Breytenbach\\
  \textbf{Date of Birth:}	& 26 June 1987	\\
  \textbf{Place of Birth:}	& Santiago, Chile\\
  \textbf{Citizenship:}		& South African\\
  \textbf{Languages:}		& English, Afrikaans (Expert); German (Intermediate) \\
  \textbf{Current Occupation:}	& PhD. Student at the South African Astronomical Observatory (SAAO) \\& and University of Cape Town, (UCT)\\
  \textbf{Work Address:}	& Room 431, RW James Building, 9 University Avenue, Upper Campus,\\& UCT, Woolsack Drive, Rondebosch, South Africa
\end{tabular}


\section{Education}
\begin{tabular}{lll}
  \textbf{2011 - present:} 	& UCT - \emph{Philosophi\ae\ Doctor} (Astrophysics)\\
    \multicolumn{3}{c}{\parbox{0.95\textwidth}{
	\begin{tabular}{p{0.175\linewidth}p{0.8\linewidth}}
	  \textbullet\ Thesis title: 	& \emph{``Rapid Variability in magnetic Cataclysmic Variable Stars''} \\
	  \textbullet\ Supervisors:	& Dr. David Buckley, A. Prof Patrick Woudt \\
	  \textbullet\ Modules: 	& Cataclysmic Variable Stars, Stellar Structures, Advanced General Relativity, Hot Topics in Cosmology, High Energy Astrophysics \\ 
	\end{tabular} }}
      \\ \\
  
  \textbf{2010:} 		& UCT - \emph{Baccalaureus Scienti\ae\ Honores} (Astrophysics and Space Science) \\
    \multicolumn{3}{c}{\parbox{0.95\textwidth}{
	\begin{tabular}{p{0.175\linewidth}p{0.8\linewidth}}
	  \textbullet\ Project title:	& \emph{``The Sferic Count Rate from SANAE-IV, Antarctica''} \\
	  \textbullet\ Supervisors:	& Dr. Andrew Collier	\\
	  \textbullet\ Modules: 	& General Astrophysics, Electrodynamics, General Relativity, Computational Astrophysics, Galaxies and Large Scale Structure, Observational Techniques, Radio Astronomy \\ 
	\end{tabular} }}
      \\ \\
  
  \textbf{2006 - 2009:} 		& University of Pretoria (UP)-  \emph{Baccalaureus Scienti\ae\ } (Physics and Astronomy) \\ %\ae\
    \multicolumn{3}{c}{\parbox{0.95\textwidth}{
	\begin{tabular}{p{0.175\linewidth}p{0.8\linewidth}}
	  \textbullet\ Project title:	& \emph{``Rutherford Backscattering Spectroscopy and X-ray Diffraction Spectroscopy of Aluminium 100''} \\
	  \textbullet\ Supervisors:	& Prof. Chris Theron	\\
	  \textbullet\ Modules: 	& Quantum Mechanics, Solid State Physics, Statistical Mechanics, Differential Calculus, Vector Calculus, Partial Differential Equations, Abstract Algebra, Mathematical Modelling \\ 
	\end{tabular} }}
      \\
      
\end{tabular}

\section{Awards and Scholarships}

% \begin{itemize}
\begin{tabular}{ll}
  \textbullet\ 2013 - present: 		& NRF, Postgraduate Development Programme (PDP) Doctoral Scholarship \\
  \textbullet\ 2010 - 2012: 		& South African Square Kilometre Array (SKA) Postgraduate Scholarship \\
  \textbullet\ 2008 - 2009: 		& SKA Undergraduate Bursary Award
\end{tabular}

 
% \end{itemize}

\section{Special Skills}

\large\textbf{Computing}
\normalsize
\vspace{0.2cm}


\begin{itemize}
  
  \item Public code repository: \hspace{1.7cm} \url{ https://github.com/apodemus }
  
  \item Languages:\\
  \begin{tabular}{p{0.5\linewidth}p{0.8\linewidth}}
    \textbullet\ Python					& (Expert) \\
    \textbullet\ IRAF, MATLAB (Octave)			& (Highly proficient) \\
    \textbullet\ IDL, R, C, Mathematica, Maple, LabView & (Intermediate) \\
    \textbullet\ SQL 					& (Novice)
  \end{tabular}

%   \begin{itemize}
%     
%   \end{itemize}
 
  \item Algorithm Development:
  \begin{itemize}
    \item Machine learning algorithms for classification of EEG data (epileptic seizure prediction)
    \item Time series analysis \& Spectral estimation techniques (Ph.D. Thesis)
    \item (Unofficial) Data reduction pipeline for Sutherland High Speed Optical Camera (SHOC)
    \item Computational modelling of Dwarf Novae in Outburst (M.Sc. Thesis)
  \end{itemize}

\end{itemize}

\vspace{0.2cm}
\large\textbf{Observing}
\normalsize
\vspace{0.2cm}
\begin{itemize}
 \item 77 nights on 1.9m telescope at SAAO, Sutherland \\
 \hspace*{0.4cm} - Rapid photometry of CVs
 
 \item 14 nights on 1.0m telescope at SAAO, Sutherland \\
 \hspace*{0.4cm} -  Multi-colour(UBVRI) photometry of interacting galaxies \\
 \hspace*{0.4cm} -  High speed photometry of CVs
 
 \item 7 nights on 1.9m telescope at SAAO, Sutherland \\
 \hspace*{0.4cm} - Spectroscopy of Dwarf Stars (SkyMapper follow-up)
 
 \item 7 nights on 1.9m telescope at SAAO, Sutherland \\
 \hspace*{0.4cm} - Polarimetry of magnetic CVs (Assisting)
 
 \item 7 nights on 1.4m Infra-red Survey Facility telescope (IRSF) at SAAO, Sutherland \\
 \hspace*{0.4cm} - Photometry of interacting galaxies (during ISYA)
 
\end{itemize}

%   \textbullet\ 77 nights on 1.9m telescope at SAAO, Sutherland \\
%   %\begin{itemize}
%     \indent - Rapid photometry of CVs \\
  %\end{itemize}
  
%   \textbullet\ 14 nights on 1.0m telescope at SAAO, Sutherland
%   \begin{itemize}
%     \item Multi-colour photometry of interacting galaxies
%     \item High speed photometry of CVs
%   \end{itemize}
  
%   \textbullet\ 7 nights on 1.9m telescope at SAAO, Sutherland
%   \begin{itemize}
%     \item Spectroscopy of Dwarf Stars (SkyMapper follow-up)
%   \end{itemize}
  
%   \textbullet\ 7 nights on 1.9m telescope at SAAO, Sutherland
%   \begin{itemize}
%     \item Polarimetry of magnetic CVs (Assisting)
%   \end{itemize}
%   
%   \textbullet\ 7 nights on 1.4m Infra-red Survey Facility telescope (IRSF) at SAAO, Sutherland
%   \begin{itemize}
%     \item Photometry of interacting galaxies (during ISYA)
%   \end{itemize}
  
% \end{itemize}

% 
\section{Professional Development}

\large\textbf{Conference Attendance}
\vspace{0.2cm}
\normalsize
\begin{itemize}
 \item 2015 September: The Golden Age of Cataclysmic Variables and Related Objects - III\\
 \hspace*{0.4cm} - Presentation: ``Quasi-Periodic Oscillations in magnetic CVs''
 \item 2015 June: SALT Science Conference 2015, STIAS, Stellenbosch \\
 \hspace*{0.4cm} - Poster: ``Probing accretion in magnetic CVs through rapid photometry with SALTICAM''
 \item 2014 July: South African Institute of Physics Conference (SAIP2014)\\
 \hspace*{0.4cm} - Presentation: ``Rapid Variability of magnetic Cataclysmic Variable Stars''
 \item 2013 September: The Golden Age of Cataclysmic Variables and Related Objects - II\\
 \hspace*{0.4cm} - Presentation: ``Modelling Quasi-Periodic Variability in Cataclysmic Variable Stars''
 \item 2013 July: SKA Joint Radio Transients Conference\\
 \hspace*{0.4cm}
 \item 2012 December: SKA Postgraduate Bursary Conference\\
 \hspace*{0.4cm} - Presentation: ``Modelling Quasi-Periodic Variability in Cataclysmic Variable Stars''
 \item 2012 August: IAU XXVIII General Assembly\\
 \hspace*{0.4cm} - Poster: ``Modelling Quasi-Periodic Variability in Cataclysmic Variable Stars''
 \item 2011 December: SKA Postgraduate Bursary Conference\\
 \hspace*{0.4cm} - Poster: ``A study of DNOs and QPOs in Cataclysmic Variable Stars''
 \item 2011 March: Middle-East and African Regional IAU Meeting II (MEARIM-II)
 \item 2010 - 2008 November: SKA Postgraduate Bursary Conference
\end{itemize}
\vspace{0.4cm}
% 
\large\textbf{Workshop Attendance}
\vspace{0.2cm}
\normalsize
\begin{itemize}
 \item 2015 April: GPGPU programming workshop, UCT
 \item 2014 November: \textsc{GADGET} Simulations workshop, UCT
 \item 2014 October: $2^{nd}$ Machine learning JEDI Workshop, Cape Town
 \item 2012 February: IAU International School of Young Astronomers (ISYA), UCT \& SAAO, Cape Town
 \item 2011 October: Workshop on Space Science and Astrophysics, Centre for High Performance Computing (CHPC), CSIR, Pretoria 
 \item 2010 December: Workshop on Convection in stars, University of the Witwatersrand (WITS), Johannesburg
 \item 2010 January: National Astrophysics and Space Science Program (NASSP) Summer School, UCT \& SAAO, Cape Town\\
\end{itemize}


% \vspace{1cm}
\section{Publications}
% \vspace{0.2cm}

\begin{refsection}[articles]
  \nocite{*}
  \printbibliography[ title={Peer-reviewed},
		      heading=subbibliography ]
\end{refsection}


\begin{refsection}[proceedings]
  \nocite{*}
  \printbibliography[title={Conference Proceedings},
		     heading=subbibliography]
\end{refsection}


\begin{refsection}[shortpub]
  \nocite{*}
  \printbibliography[title={Short publications},
		     heading=subbibliography]
\end{refsection}


% \begin{refsection}[proceedings]
%   \nocite{*}
%   \printbibliography[title={Conference Proceedings},
% 		     heading=subbibliography]
% \end{refsection}

% \large\textbf{Short Publications}
% \normalsize\vspace{0.2cm}



 
 

\section{Work and Teaching Experience}
% \vspace{0.2cm}
\begin{itemize}
 \item 2013: Tutor: Stellar Structures (AST3002F), UCT
 \item 2012: Tutor: Introductory Astronomy (AST2002S), UCT
 \item 2012: Tutor: Introductory Astronomy (private)
 \item 2011: Tutor: Electrodynamics Honours, NASSP, UCT
 \item 2009: Tutor: Biological Physics (PHY103), $1^{st}$ year course, UP
 \item 2007 Dec - Jan: Student Data Processor, Hartebeest Hoek Radio Astronomy Observatory (HartRAO)\\
\end{itemize}



\section{Research interests}
% \vspace{0.2cm}
\begin{itemize}
 \item Astrophysical Accretion on all scales
 \item Compact binary stars
 \item Machine Learning
%  \item Multi-wavelength
%  \item Bayesian Statistics
 \item Computational Modelling
\end{itemize}



\section{Leadership \& Extracurricular Involvements}
\large\textbf{Academic}
\vspace{0.2cm}
\normalsize
\begin{itemize}
 \item 2014-2015: Postgraduate Student Representative for Astronomy Dept., UCT
 \item 2014: Chairperson: UCT Mountain and Ski Club
 \item 2011 - present: Committee member: UCT Mountain and Ski Club
 \item 2011: Siyavula Education Program
 \begin{itemize}
  \item[-] Tasks: Proof reading and Translation into Afrikaans of open source High School science textbooks
  \end{itemize}
\end{itemize} 
\vspace{0.2cm}

\large\textbf{Sport}
\vspace{0.2cm}
\normalsize
\begin{itemize}
  \item 2012: UCT Sports Merit Award
  \begin{itemize}
      \item[-] UIAA Youth expedition to summit the highest mountain in Europe, Mt Elbrus at 5 642m
  \end{itemize}
  \item 2011: UCT Sports Performance of the Year Award
  \begin{itemize}
      \item[-] UCT Mountain and Ski Club (MSC) mountaineering expedition to summit the 6 264m Himalayan peak, CB13-A	
  \end{itemize}
 
\end{itemize}


\section{References}
\begin{itemize}
 \item Emeritus Distinguished Professor Brian Warner\\Astronomy Department\\University of Cape Town\\Phone: +27 21 650 2391\\email: brian.warner@uct.ac.za
%  \item 
 \item Professor Chris Theron\\Head of Department Physics\\ University of Pretoria\\Phone +27 12 420 2455\\email: chris.theron@up.ac.za
\end{itemize}
% 
%  
%  
%  
%  
%  
%  
%  
%  
% \end{resume}
\end{document}
